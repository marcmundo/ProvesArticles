MECHATRONICS, according to one of its most commonly
used definitions, is ―the interdisciplinary field of engineering
dealing with the design of products whose functions rely on
the integration of mechanical and electronic components
coordinated by a control architecture‖ [1]-[4]. The first
European university to offer a degree in mechatronics
engineering was the University of Lancaster (United
Kingdom) in 1984. The degree was a response to industry’s
need for engineers with a multidisciplinary profile [5]. Later,
in the second half of the eighties, these studies were offered in
other universities at both undergraduate and postgraduate
levels [6], [7], [16], and became particularly relevant in
European countries with strong industry such as Germany,
United Kingdom or Finland or USA, Mexico, Brazil, Japan or
Australia among others around the world. In the 2009/2010
academic year the University of Vic (UVic) began a degree in
mechatronics engineering [8], [9] in accordance with the new
European Higher Education Area (EHEA) framework, and has
thus become a pioneer of these studies in Spain. Like the first
European mechatronics degree, this degree is also the result of
collaboration between the university, in this case UVic, and
local industrialists, who participated in designing the
curriculum. The curriculum includes a module called
Integrated Projects (IP) in which students have to combine
multidisciplinary knowledge (mechanics, electronics, control
and computer science) acquired in different subjects studied
previously. In order to create a system that integrates all these
mechatronics disciplines [11] a six-axis robotic arm has been
designed (Alpha6UVic) to be used in practical classes in this
module. Comparing this educational project with others
existing in the literature, it can be seen, for example, that [10]
is a project aimed more at industrial automation than
mechatronics, as the mechanical part is fixed. In the same way
the mechanics are fixed in other educational projects [17]-[20]
that use mobile robots based on kits. In these projects there is
a lot of programming which makes them adequate to the study
of sensors but limited for learning mechanics. In the case of
[21] the project is much more fully developed but focuses
more on automotive industry. By contrast the focus of the
project described here was to reflect the needs of the industrial
sectors in the local context. The aim of these kinds of projects
is for students to be able to work on different multidisciplinary
projects. The novelty of using the Alpha6UVic is that it is an
open mechatronic project, where the technology can be
completely modified from start to finish. This allows the
teacher to present a case study based on the Alpha6UVic and
the students of each course to propose various mechatronic
solutions and apply these changes to it. In contrast, if a
commercial robot were used, it would not be possible to
modify it and neither would the necessary technical
information be available to carry out the modifications. As
skills and knowledge from different interrelated disciplines are
necessary to develop a robot we consider that this project fits
the mechatronics paradigm very well. This paper presents the
main features of the robot and how it has been used in the
practical classes in the IP module. Section II introduces the
six-axis robot arm with information ordered in nine parts:
mechanics, sensors, actuators, power supply, control boards,
I/O board, central unit, communication bus and the shell.
Section III is devoted to the educational aspects and describes
the utilization of the robot in the IP module. Section IV
outlines the evaluation of IP and the use of Alpha6UVic
within this module. The conclusions are reported in section V.

